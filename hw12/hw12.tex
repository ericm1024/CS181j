\documentclass[11pt]{article}

\usepackage[colorlinks=true]{hyperref}

% This is a toggle for whether the solutions should be included in output of this document
\newif\ifSolutions
%\Solutionsfalse  % this is used to exclude the solutions
\Solutionstrue  % this is used to include the solutions

\input{../latexDefinitions}

\head{12}{Friday, Dec 4, 2015}{Friday, Dec 11, 2015}

In this homework assignment, you'll do some examples of using basic \href{http://www.open-mpi.org/doc/v1.8/}{MPI} features.  
To warm up, you'll parallelize the simple scalar integrator, which gives you practice with collectives.
Next, you'll parallelize the wave equation, which gives you practices with point-to-point communication.
Next, you'll visit Mordor for an exercise that helps you think about communication patterns.  
Finally, you'll be implementing a communication pattern that is common in large-scale scientific codes.

Blah blah, \texttt{Main0.cc}, example syntax, yadda yadda.

On this assignment, we won't actually be doing any careful timing or speedups because we only have shared memory machines on which to run, which is a bummer.  
You won't have to worry about ``running at the same time'' as other people because it shouldn't affect your results at all, but feel free to run on \texttt{knuth} if you feel like 12 cores isn't enough for you (first world problems!).

\textbf{Note:} Friendly PSA: if you want to send the contents of a \texttt{vector\textless Something\textgreater  v} with MPI, you need to pass a pointer to the first entry in \texttt{v}, which is \texttt{\&v[0]}, \textbf{NOT} \texttt{\&v}.

\eprob{5}{Strong scaling is hard}

It is often quite difficult to achieve good scaling in parallel programs.  
That is, if you run a parallel program on $n$ processors, it is hard to get it to go $n$ times as fast.  
I often hear some dubious claims about speedups on the order of thousands, which is enough to raise eyebrows.

Suppose that you sit down and think about your algorithm and determine that $s$\% of it must be done in serial.  
You have access to an enormous supercomputer and have no limit to the number of processors you can use.  
Parallel portion runtimes scale linearly with how many processors used.  
Ignore cores for now.

Derive an expression for the theoretical speedup possible for an algorithm that is $s$\% inherently serial run on $p$ processors, where the parallel sections scale linearly.  
Plot this expression versus $p$ for various values of $s$.  
What is required to get speedups on the order of thousands?  
What are some aspects of parallel computing that might keep you from getting this type of speedup?

\ifSolutions
\textbf{Solution:}

\begin{equation}
  \text{speedup}=\frac{\pi}{e} > 9000
\end{equation}

{
  \begin{center}
  \includegraphics[width=4.0 in]{StrongScaling}
  \end{center}
}

\fi

\vfill








\eprob{5}{Integration with Messages}

Parallelize the scalar function integrator.  
What's the minimum problem size you have to be doing to get any appreciable speedup? 
For the largest number of intervals ($10^9$), how well does your program scale versus the number of processes?

\ifSolutions
\textbf{Solution:}

\begin{center}
  \begin{tabular}{ | c | c | c |}
    \hline
    Number Of Processes  & Runtime & Speedup     \\ \hline \hline
    1  & TODO & 1.0     \\ \hline
    6  & TODO & TODO \\ \hline
    12 & TODO & TODO \\ \hline
    24 & TODO & TODO \\ \hline
  \end{tabular}
\end{center}

\fi

\vfill









\estretchprob{0}{15}{K Means I'll Never Do This Again}

Alright, so you've vectorized it, threaded it, and GPU-ized it - it's time to distributed-memory-parallelize it.  
Parallelize the K-Means clustering calculation.
Don't worry about going fast, just get the same answer (i.e., you should be able to run it with ``\texttt{mpirun -np 12 ./KMeansClustering}'' and it should complete successfully).

\ifSolutions
\textbf{Solution:}

Well, uh, does it work?

\fi

\vfill










\eprob{10}{Making Distributed Waves}

Parallelize the provided serial equation solver.  Don't worry about any fancy tricks to reduce communication, just communicate on every timestep.

\begin{enumerate}[a)]
\item Include a couple of plots (boo!) or a link to a movie (yay!) that shows a waveform moving through the domain.  
  
\ifSolutions
\textbf{Solution:}

I love this \href{https://www.youtube.com/watch?v=cvCjyWp3rEk}{video} so much.

\fi

\item More or less, how well does the parallel wave equation solver scale compared to the threaded solver's scaling?

\ifSolutions
\textbf{Solution:}

\fi

\end{enumerate}

\textbf{Note:} In order to postprocess the data, you'll need to add an argument for the number of processes you used in the simulation: \texttt{``python generate1DFDWaveEquationPlots\_mpi.py 24''} and \texttt{``./MakeMovie 24''}.  
The figures are colored by the rank that wrote each piece of the domain.

\vfill












\eprob{20}{Frodo's Chocolates}

Frodo is invited to Saruman's variant of a white elephant gift exchange where players each bring a gift, sit in a circle, and pass their gifts some number of times randomly either all to the left or all to the right.  
Because Frodo brought chocolates to the party and everyone else brought old homework assignments, Frodo's gift is highly desired.  
The party is very long: exchanges will be made some large number of times (\texttt{maxExchanges}, like 10000).  
However, the participants are impatient and, if one participant is passed the delectable chocolates some other smaller number of times (\texttt{rageQuitLimit}, like 100), the person will ragequit and run away with them.

Perform a simulation modeling this gift exchange.  In particular,

\begin{itemize}
\item Each rank will start with a gift and keep track of the number of times it ends up with Frodo's gift.
\item On each one of the \texttt{maxExchanges} exchanges, all ranks pass their current gift either to the right or to the left. 
The right neighbors of ranks $0$, $4$, and $n-1$ are $1$, $5$, and $0$ respectively. 
The left neighbors of ranks $0$, $4$, and $n-1$ are $n-1$, $3$, and $n-2$ respectively.

\textbf{Note:} Friendly PSA: -1 \% 3 is not 2.

\item The game ends when either the max number of allowed exchanges (\texttt{maxExchanges}) have been made or a rank has held Frodo's chocolates \texttt{rageQuitLimit} times.
\item Once the game has ended, use the provided script to make a bar graph showing the number of times each rank held Frodo's chocolates.
\item The results must be different each time you run the program (i.e. don't fix the random seed).
\end{itemize}

\begin{enumerate}[a)]
\item Just for funsies, what do you expect the final distribution of chocolate-holding frequency to be?  
How will it be affected by \texttt{maxExchange}, \texttt{rageQuitLimit}, and the total number of processes?

\ifSolutions
\textbf{Solution:}

\fi

\item List two ways in which you can make sure that all ranks are passing in the same direction at the same time.  
Be specific: what function calls would you make, who makes them, etc.?

\ifSolutions
\textbf{Solution:}

\fi

\item How do you make sure that the game ends gracefully and the distribution of chocolate-holding is correctly compiled when one player quits?

\ifSolutions
\textbf{Solution:}

\fi

\item Implement the simulation. 
Show some (more than one) final chocolate-holding frequency distributions for some interesting combinations of \texttt{maxExchange}, \texttt{rageQuitLimit}, and the total number of processes.  
Were your predictions in part a correct?

\ifSolutions
\textbf{Solution:}

{
  \begin{center}
  \includegraphics[width=3.0 in]{figures/FrodosChocolates1}
  \end{center}
}

\fi

\end{enumerate}

\vfill











\eprob{40}{Funball aftermath}

After a particularly raucous funball, the students wake up the following morning to find that there is money scattered everywhere (among other things). 
In this problem, each rank represents a dorm, and each dorm houses students with IDs that are unique within that dorm. 
Just like in real life, all the money that's scattered around is marked with the owner's dorm and the owner's ID within that dorm.
At least one \texttt{MuddMoney} exists for each student.

Write the communication that returns all of the money to its proper owners.
This outline might help you (or not):

\begin{verbatim}
initialize this rank's MuddMoneys
store all of the MuddMoneys which belong to students in this dorm (rank)
send all the MuddMoneys which belong to students from other dorms to their dorms
receive MuddMoneys that belong to your students
add received MuddMoney values to your students' MuddMoneys
check answer
\end{verbatim}

Some notes:

\begin{itemize}
  \item You are a dorm. 
You house some fixed number of students \texttt{N}. 
After funball, you find some number of Mudd moneys \texttt{M} where \texttt{M}$>$\texttt{N}. 
You are guaranteed to start with one mudd money per student in your dorm.
You also have some other mudd moneys that belong to students in other dorms. 
  \item After you sort everything out, you need to have a vector of MuddMoneys, one per student in your dorm, with the total money that belongs to each student. 
In other words, you'll need to sum moneys that you receive into the moneys you have.
The vector does not have to be in any particular order to pass the check.
  \item The ID of each student in a dorm is unique \textit{within} that dorm, but their IDs are \textit{not} globally unique.
For example, I think that every dorm has a student 0.
  \item The IDs of the students in each dorm are not contiguous; each dorm may have 1000 students, but one student ID may be 1234567.
  \item There is too much total money to bring all of it to one rank with a \texttt{Gather} or \texttt{Allgather}.  
You may use collective operations (such as \texttt{Allreduce} or \texttt{Gather}) to \emph{coordinate} between the ranks, but \textbf{MuddMoney must only be communicated in point-to-point communication}.
  \item If rank 7 has to send 8 moneys to rank 3, it should only send one message.  
Don't send each individual MuddMoney in a separate message.  
That is lunacy.
  \item Each dorm (rank) will only need to send information to a subset of all of the dorms, not to everyone.  
For example, there may be $50$ total ranks, but rank $5$ is only going to need to send information to maybe $8$ other ranks.  
Except for any collective operations you use to coordinate, each rank should only do point-to-point communication with the ranks it \textit{has to} send information to, not to every rank out there.
That would also be lunacy.
  \item You are welcome to change the random seed to a fixed number for debugging purposes, but for grading, your code will be tested with a random seed and several different numbers of ranks (like 5, 10, 25, and 50).
Please commit your code with a random seed, or grutors will be sad and cranky.

\end{itemize}

\ifSolutions
\textbf{Solution:}

Well, uh, did it work?

\fi

\vfill












\eprob{15}{Back to the Real World}

\begin{enumerate}[a)]
\item Explain as you would to a ``non-technical'' friend of yours what a cluster is.
  
\ifSolutions
\textbf{Solution:}

\fi

\item Explain as you would to a ``non-technical'' friend of yours what distributed-memory parallel programming is and how MPI fits into anything.
  
\ifSolutions
\textbf{Solution:}

\fi

\item Explain as you would to a ``non-technical'' friend of yours the difference between blocking and non-blocking communication.
  
\ifSolutions
\textbf{Solution:}

\fi
  
\item Suppose your lab-mate hears that MPI make codes go really fast and asks for your help in parallelizing their code.  
What kind of questions would you ask them to figure out if parallelizing their code is a good idea or even possible?  
  
\ifSolutions
\textbf{Solution:}

\fi

\end{enumerate}

\eprob{5}{Feedback}

\begin{enumerate}[a)]
\item How much total time did you spend on this assignment?
\item Of the total time, how much total time did you spend ``flailing'' on little annoying things that are not the main point of the assignment?
\item Did you have any ``aha'' moments where something clicked?  If so, on what problems or parts?
\item Can you give me any feedback on this assignment?
\item If you want to, add a cat picture (or some other picture that would be fun) to use on assignments the next time I teach this.
\begin{center}
    \includegraphics[width=4.0 in]{JamesSaindonStretchProblems}
\end{center}
\end{enumerate}

\vfill

\vskip 1cm
\total

\end{document}

todo: prepare the papi-ed, bucketed, vectorized, threaded, gpu-ized, and distributed-memory-parallelized matrix multiplication problem for the last homework.  stretch points for incorporating the communication avoidance from the colloquium.
